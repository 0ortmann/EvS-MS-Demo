% !TeX root=../paper.tex
\section*{Addendum}

Unter \url{https://github.com/0ortmann/EvS-MS-Demo} bieten wir eine Referenzimplementation eines Microservice Systems zu illustrativen Zwecken. Es werden vier unterschiedliche Microservices verwendet, die miteinander interagieren. Drei der Services sind selbst implementiert. Ein autonomes Frontend, realisiert mit \textit{React}\footnote{\url{https://facebook.github.io/react/}}, ein Bildverarbeitungs-Backend in Python und ein Service der das Speichern von Bildern und Anmelden von Benutzern handhabt, geschrieben in PHP. Als Speicherlösung -- die als vierter Microservice verstanden wird -- verwenden wir eine \textit{MongoDB}\footnote{\url{https://www.mongodb.org/}}. Unsere Referenzimplementation verwendet unter anderem \hyperref[subsubsec:jwt]{\em JSON Web Tokens}. Dockerfiles stehen zur Verfügung, um die einzelnen Services in Container zu kapseln.

Im Zusammenspiel bieten all diese Microservices ein System, mit dem es einem Nutzer möglich ist, über eine Webpage im Browser PNG-Bilder hochzuladen und diverse Algorithmen zur Kantenerkennung darauf anzuwenden. Das anschließende Ergebnis wird ebenfalls in Bildform in der Datenbank gespeichert und der Nutzer erhält die Möglichkeit, es im Browser anzuschauen oder herunterzuladen.

Das Beispiel macht deutlich, dass nur im Zusammenspiel aus all den Services ein sinnvolles Gesamtsystem entsteht. Weiterhin wird klar, wie eine Authentifizierung via JWT umgesetzt werden kann. Dieses Beispiel ist kein \textit{Boilerplate}, sondern eine alleinstehende Referenzimplementation im Zuge dieser Ausarbeitung.
