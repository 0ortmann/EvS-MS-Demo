\section{Schutzziele}
Nachdem wir bisher auf die Architektur und die Kommunikation von Microservices eingegangen sind, betrachten wir im folgenden Abschnitt Angriffspunkte einer solchen Infrastruktur. Explizit werden drei Schutzziele der Informationssicherheit betrachtet, welche eine besondere Relevanz in verteilten Systemen spielen. Zunächst gehen wir auf das der Vertraulichkeit in \autoref{subsec:vertraulichkeit} ein. Weitergehend setzt sich \autoref{subsec:integrität} mit der Integrität und zuletzt \autoref{subsec:verfügbarkeit} mit der Verfügbarkeit von Informationen innerhalb eines Microservice-Netzwerks auseinander.


\subsection{Vertraulichkeit}
\label{subsec:vertraulichkeit}

Zur Einführung der drei Schutzziele ziehen wir die Definitionen von \citeauthor{Bedner+10} zurate, um auf diese näher eingehen zu können. Zunächst widmen wir uns der Vertraulichkeit.

„Informationsvertraulichkeit ist bei einem IT-System gewährleistet, wenn die darin enthaltenen Informationen nur Befugten zugänglich sind. Dies bedeutet, dass die sicherheitsrelevanten Elemente nur einem definierten Personenkreis bekannt werden. Dazu sind Maßnahmen zur Festlegung sowie zur Kontrolle zulässiger Informationsflüsse zwischen den Subjekten des Systems nötig (Zugriffsschutz und Zugriffsrechte), sodass ausgeschlossen werden kann, dass Informationen zu unautorisierten Subjekten ‚durchsickern‘.“ \cite{Bedner+10}

Für eine Microservice-Architektur betrifft dies vor allem die Kommunikation zwischen den einzelnen Microservice-Instanzen. Um diese für beide Gesprächspartner abzusichern werden zwei verschiedene Verfahren vorgestellt, zum einen die Handshake-basierte Authentifikation, bei der sich beide Gesprächspartner einander vorstellen, und die Token-basierte Authentifikation, bei welcher ein Geheimnis, ein sogenanntes „Token“, die Inhalte der Kommunikation quittieren und belegen.

\subsubsection{Handshake-basierte Authentifikation}

\subsubsection{Token-basierte Authentifikation}


\subsection{Integrität}
\label{subsec:integrität}
„Integrität oder Unversehrtheit bedeutet zweierlei, nämlich die Vollständigkeit und Korrektheit der Daten (Datenintegrität) und die korrekte Funktionsweise des Systems (Systemintegrität). Vollständig bedeutet, dass alle Teile der Information verfügbar sind. Korrekt sind Daten, wenn sie den bezeichneten Sachverhalt unverfälscht wiedergeben. Die Integrität bedeutet, dass Daten im Laufe der Verarbeitung oder Übertragung mittels des Systems nicht beschädigt oder durch Nichtberechtigte unbefugt verändert werden können. Beschädigungs- oder Veränderungsmöglichkeiten sind das Ersetzen, Einfügen und Löschen von Daten oder Teilen davon.“ \cite{Bedner+10}

\subsubsection{Verschlüsselungsverfahren}


\subsection{Verfügbarkeit}
\label{subsec:verfügbarkeit}
„Die Verfügbarkeit betrifft sowohl informationstechnische Systeme als auch die darin verarbeiteten Daten und bedeutet, dass die Systeme jederzeit betriebsbereit sind und auf die Daten wie vorgesehen zugegriffen werden kann. Zum einen muss die Datenverarbeitung inhaltlich korrekt sein und zum anderen müssen alle Informationen und Daten zeitgerecht zur Verfügung stehen und ordnungsgemäß verarbeitet werden.“ \cite{Bedner+10}

\subsubsection{Redundanz und Lastverteilung}

\subsubsection{Clientseitige Lastverteilung}

In dem Blog-Beitrag \cite{Li15} wird ein neuartiger Ansatz motiviert, welcher die Verfügbarkeit von Lastverteilung verbessert. Dies ist notwendig, da der Last-verteilende \textit{Load Balancer} einen Flaschenhals darstellt. Jegliche Serverzugriffe werden zunächst im anvertraut. Dies stellt einen sogenannten \textit{„Single Point of Failure“} dar, denn sollte der \textit{Load Balancer} ausfallen, ist das System nicht mehr erreichbar.

\begin{figure}
	\caption{Schematische Darstellung von clientseitiger Lastverteilung}
	\label{fig:clientseitige_lastverteilung}
\end{figure}

Abhilfe für dieses Problem stellt der folgende clientseitige Lastverteilungsdienst dar. Diese funktioniert so, dass jeder Microservice, welcher auf die Dienste anderer angewiesen ist, über seine eigene Lastverteilung verfügt. Ein solcher Aufbau ist in \autoref{fig:clientseitige_lastverteilung} schematisiert: Ein Microservice verfügt über seinen eigenen \textit{Load Balancer}, an welchen er seine ausgehende Kommunikation exklusiv leitet. Dieser kennt alle anderen Microservice-Instanzen und kann daher Nachrichten an diese weiterleiten.

Die von \citeauthor{Li15} entwickelte Grundstruktur \textit{„Baker Street“}\footnote{\textit{Baker Street} ist unter Apache-Lizenz 2 veröffentlichte freie Software, siehe \url{http://bakerstreet.io/}.} ist eine Umsetzung dieser Architektur. Kern dieser sind drei folgenden Komponenten.

\begin{itemize}
	\item Der \textit{HAProxy}-basierte \textit{Load Balancer} \textbf{Sherlock} welcher lokal jeder Microservice-Instanz beiwohnt. Er findet heraus, wohin Verbindungen von dieser Instanz wandern sollen.
	
	\item Der \textit{Health Checker} \textbf{Watson} über den ebenfalls jede lokale Instanz verfügt. Er übermittelt den Zustand des Microservices und meldet diesen ebenfalls initial an.
	
	\item Zuletzt gibt es das \textbf{Datawire-Verzeichnis}, ein leichtgewichtiger globaler Diensterkundungs-Mechanismus, welcher Informationen von jeder Watson-Instanz erhält und Verfügbarkeitsänderungen zu jeder lokalen Sherlock-Instanz weiterleitet.
\end{itemize}

\subsubsection{Kontinuierliches und unabhängiges Deployment}