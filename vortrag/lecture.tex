% !TeX program = xelatex
% !TeX spellcheck = de_DE
\documentclass{beamer}
\usepackage{fontspec}
\usepackage{polyglossia}
\usepackage[notocbib]{apacite}
\usepackage{hyperref}
\usepackage{xspace}
\usepackage{array}
\usepackage{fontawesome}

\setsansfont{Calibri}
\setdefaultlanguage{german}
\usetheme{UHH}

\def\authors#1#2{\author[#1]{#2}}
\def\email#1{\texttt{#1}}
\title{Microservices und Sicherheit}
\subtitle{Seminar Komponenten, Agenten und Workflows in verteilten Systemen}
\authors{Felix Ortmann \& Konstantin~Möllers}{Felix~Ortmann und Konstantin~Möllers\\ \email{\{0ortmann,1kmoelle\}@informatik.uni-hamburg.de}}
\institute{Universität Hamburg\\Fachbereich Informatik\\Vogt-Kölln-Straße 30\\22527 Hamburg}


\date{16. Januar 2016}
\titlegraphic{\includegraphics[width=3cm]{img/uhh.pdf}}

\renewcommand{\APACrefauthstyle}{\bfseries}
\setcounter{tocdepth}{1}
\def\tocname{Gliederung des Vortrags}
\AtBeginSection{\frame{\frametitle{\tocname}\tableofcontents[currentsection]}}

\newcolumntype{x}[1]{>{\centering\arraybackslash}m{#1}}
\setbeamertemplate{headline}[default]

\defbeamertemplate*{title page}{customized}[1][]
{
	{\color{leuchtrot}
	\usebeamerfont{title}\centering\inserttitle\par
	\centering\usebeamerfont{subtitle}\insertsubtitle\par}
	\bigskip
	\centering\usebeamerfont{author}\insertauthor\par
	\bigskip
	\usebeamerfont{institute}\insertinstitute\par
	\bigskip
	\usebeamerfont{date}\insertdate\par
}

\begin{document}

\fontsize{14pt}{14pt}
	
% Titelfolie
{\setbeamercolor{title page}{bg=leuchtrot}
\begin{frame}%
	\titlepage
\end{frame}}

% Gliederung
\begin{frame}{\tocname}
	\tableofcontents
\end{frame}

\section{Einführung}

\subsection{Eigenschaften von Microservices}
\begin{frame}{\insertsubsection}
	\begin{itemize}
		\item \textbf{Kohäsion} „gleiches zu gleichem!“
		\item \textbf{Autonomie} „le Service c'est moi!“
		\item \textbf{Kooperation} „toll, ein anderer macht's!“
	\end{itemize}
\end{frame}

\subsection{Abgrenzung}
\begin{frame}{\insertsubsection}
	\begin{itemize}
		\item \textbf{SOA}\\
		fehlt: gekapseltes Frontend.\\
		hier: klare Trennung der Systeme, generische Schnittstelle durch das Web.
		
		\item \textbf{MAS}\\
		fehlt: soziologische Theorie.\\
		hier: soziotechnische Perspektive.
		
		\item \textbf{MOrgaS} \cite{Wester-Ebbinghaus10} \\
		fehlt: Schachtelbarkeit und Hierarchie.\\
		hier: Föderalismus, dezentralisierte Governance.
	\end{itemize}
\end{frame}

\definecolor{greenc}{HTML}{00AB84}
\definecolor{redc}{HTML}{EF3340}

\def\Plus{\color{greenc}+}
\def\Minus{\color{redc}–}
\subsection{Für und Wider}
\begin{frame}{\insertsubsection}
	\begin{columns}
		\begin{column}{.5\linewidth}
			\centering\textbf{\textcolor{greenc}{Pro}}
			\begin{itemize}
				\item[\Plus] Skalierbarkeit auf Enterprise-Ebene
				\item[\Plus] höhere Entscheidungsbefugnis für Entwickler
				\item[\Plus] \textit{Rapid Prototyping} von Anwendungen möglich
				\item[\Plus] schnellerer Zugang zu neuen Technologien und Programmiersprachen
			\end{itemize}
		\end{column}
		\begin{column}{.5\linewidth}
			\centering\textbf{\textcolor{redc}{Kontra}}
			\begin{itemize}
				\item[\Minus] hoher technischer Aufwand
				\item[\Minus] ausführliche Absprache und gute Tests obligatorisch
				\item[\Minus] Latenz und Overhead durch Kommunikation
				\item[\Minus] übergreifendes System-Refactoring schwer bis unlösbar
			\end{itemize}
		\end{column}
	\end{columns}
	
\end{frame}

\subsection{Motivation: Microservices + Sicherheit}
\begin{frame}{\insertsubsection}
	\begin{itemize}
		\item unabhängige, kleine Teams
		\begin{itemize}
			\item[$\Rightarrow$] bessere, leichtere Kommunikation
			\item[$\Rightarrow$] effizientes Debugging und Pair-Programming
		\end{itemize}
		\item Unabhängigkeit der Ausführung von Services
		\begin{itemize}
			\item[$\Rightarrow$] Hoheit über eigene Software
			\item[$\Rightarrow$] rasches, häufiges Deployment und schließen von Sicherheitslücken
		\end{itemize}
		\item einfache, generelle Schnittstelle
		\begin{itemize}
			\item[$\Rightarrow$] viele gut getestete Tools für die Kommunikation
			\item[$\Rightarrow$] Skalierung leichter und schneller möglich
		\end{itemize}
	\end{itemize}
\end{frame}

	
\section{Microservice-Architekturen und Sicherheit}
\section{Kommunikation und Sicherheit}
\section{Schutzziele}

\subsection{Vertraulichkeit}
\begin{frame}{\insertsubsection}
	\textbf{Informationsvertraulichkeit} ist bei einem IT-System gewährleistet, wenn die darin enthaltenen Informationen \textbf{nur Befugten} zugänglich sind. Dies bedeutet, dass die sicherheitsrelevanten Elemente nur einem \textbf{definierten Personenkreis} bekannt werden. Dazu sind \textbf{Maßnahmen} zur Festlegung sowie zur Kontrolle zulässiger Informationsflüsse zwischen den Subjekten des Systems nötig (\textbf{Zugriffsschutz und Zugriffsrechte}), sodass ausgeschlossen werden kann, dass Informationen zu unautorisierten Subjekten \textbf{„durchsickern“}.
\end{frame}
\subsection{Verfügbarkeit}
\subsection{Integrität}

%http://jwt.io/introduction/

\section{Zusammenfassung}

% Literatur
\section{\bibname}
\begin{frame}[allowframebreaks]{\bibname}
	\AtBeginSection{}
	\nocite{*}
	\bibliographystyle{apacite}
	\bibliography{bib}
\end{frame}
\begin{frame}[allowframebreaks]{Bildquellen}
\end{frame}

\section{Demo und Diskussion}

\end{document}