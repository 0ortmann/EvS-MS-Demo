\section{Microservice-Architekturen und Sicherheit}

Ein Microservice-basiertes System ist ein verteiltes System, das in der Regel auf mehreren Maschinen läuft. 

Es gibt verschiedene Möglichkeiten, Microservices mit einander interagieren zu lassen um daraus ein komplexeres System entstehen zu lassen. Generell gelten für Microservices, dass sie \textit{klein}, auf \textit{eine Aufgabe} spezialisiert und \textit{autonom} sind \cite{newman2015building}. 


\subsection{Prozessbasierte Microservice}

\subsection{Technologien zum Bau von Microservices}

\subsubsection{Virtuelle Maschinen}

Ein Microservice sollte replizierbar (und somit skalierbar), sowie einfach zu deployen und sicher sein \cite{newman2015building,microservicesIO}. Um Skalierbarkeit und einfaches Deployment zu erreichen, bietet es sich wiederum an Microservices in ähnlichen Ausführungsumgebungen laufen zu lassen. Das jedoch steht im Wiederspruch zu dem Konzept der Service-Unabhängigkeit und Autonomie. 

Wenn mehrere Microservices den gleichen Host oder die gleiche virtuelle Maschine (\textit{VM}) teilen, sind sie nicht unabhängig. Bei böswilliger Manipulation oder Infiltrierung eines Services könnten auch andere Services im gleichen System beeinflusst werden. Folgt man dem Konzept so wäre es korrekt und sicher(-er) einen Service pro Host oder VM zu deployen. Das resultiert allerdings in Resourcen-Overhead der benötigt wird, um die (möglicherweise vielen) virtuellen Maschinen bzw. Hosts zu betreiben \cite{microservicesIO}.

\subsubsection{Container}
Um diese Probleme zu lösen, bieten sich Container an. Ein Container ist ein schlanker \glqq Wrapper\grqq um einen Linux Prozess. Wir betrachten hier Docker (\url{https://www.docker.com}) als Container Engine, es sei aber noch LXC (Linux Containers) erwähnt.

Docker und Docker-Container bieten ähnliche Funktionen wie eine virtuelle Maschine und zugehörige VM Images. Allerdings sind Container um ein Vielfaches leichtgewichtiger als VMs. Während eine VM ein Gast-Betriebssystem und eigene Libs enthält, enthalten Container nur die Anwendung und deren benötigte Abhängigkeiten. Container teilen sich den Kernel des ausführenden Host-Betriebssystems und laufen als einzelner Prozess im Userland \url{https://www.docker.com/what-docker}. 

Zusätzlich werden Container durch Container Images spezifiert -- genauso wie bei VMs -- was die Verfielfältigung ungeheuer einfach macht. Allerdings ist die Modifikation von Dockerimages einfacher als das Ändern eines VM Images. Docker ermöglicht mithilfe eines sogenannten \textit{Dockerfile}s sehr einfache Manipulation von Images.

Allein die Charakteristika von Docker-Containern legen nahe, wie gut diese Technologie geeignet ist um die Anforderungen, die eine Microservice-Architektur mit sich bringt, abzudecken. Durch das Design von Docker sind in Containern laufende Prozesse vollständig unabhängig voneinander. Services lassen sich in Form von Images einfach replizieren und portieren. Das ausführende System ist unwichtig, solange es ein Linux System ist und Docker installiert ist. 
\subsection{Managing Microservices / \glqq Orchestration\grqq}