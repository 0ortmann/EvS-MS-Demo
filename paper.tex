% !TeX spellcheck=de_DE
% !TeX program=xelatex
\documentclass[oribibl,utf8]{llncs}
\usepackage{hyperref}
\usepackage[ngerman]{babel}
\usepackage[utf8]{inputenc}
\usepackage[T1]{fontenc}
%\usepackage{fontenc}
\usepackage[notocbib]{apacite}
\usepackage{array}
\usepackage{amsmath, amsfonts, amssymb}
\usepackage[a4paper, hmargin=2.5cm, vmargin=2.0cm]{geometry}
\usepackage{enumerate}
\usepackage{xspace}
\usepackage{graphicx}
\usepackage{xcolor}
\usepackage{enumerate}
\usepackage{fontawesome}
\usepackage{subfigure}
\usepackage{wrapfig}
\def\authors#1#2{\authorrunning{#1}\author{#2}}
\title{Microservices und Sicherheit}
\subtitle{Seminar Komponenten, Agenten und Workflows in verteilten Systemen}
\authors{Felix Ortmann \& Konstantin~Möllers}{Felix~Ortmann und Konstantin~Möllers\\ \email{\{0ortmann,1kmoelle\}@informatik.uni-hamburg.de}}
\institute{Universität Hamburg\\Fachbereich Informatik\\Vogt-Kölln-Straße 30\\22527 Hamburg}



\newcolumntype{x}[1]{>{\centering\arraybackslash}m{#1}}

\def\calibri#1{\begingroup\fontspec{Calibri}\selectfont#1\endgroup}

\newcommand{\subfigureautorefname}{Abb.}
\renewcommand{\figureautorefname}{Abbildung}
\renewcommand{\sectionautorefname}{Abschnitt}
\renewcommand{\subsectionautorefname}{Unterabschnitt}

\linespread{1.1}

\usepackage{scrpage2}
\pagestyle{scrheadings}
\clearscrheadings
\ofoot{\pagemark}

\begin{document}
\maketitle
\begingroup
\let\oldclearpage\clearpage
\let\clearpage\relax

\begin{abstract}
	% Motivation

	
	% Zu lösendes Problem

	
	% Lösungsansatz

	
	% Ergebnisse

\end{abstract}

\section{Einführung}
\label{sec:einführung}

Zunächst führen wir allgemein in das Prinzip der Microservices ein. In diesem Abschnitt werden wir in \autoref{subsec:funktionsweise} darauf eingehen, wie die Funktionsweise von Microservices beschrieben werden kann. Daraufhin grenzt \autoref{subsec:abgrenzung} ab, wie sich diese von anderen Modellen unterscheiden. Andererseits motivieren wir den Gegenstand dieser Arbeit, indem wir in \autoref{subsec:motivation} bestimmte Eigenschaften von Microservices hinsichtlich der Sicherheit untersuchen.

\subsection{Eigenschaften von Microservices}
\label{subsec:funktionsweise}

Um die Funktionsweise von Microservices zu erläutern, betrachten wir die Definition nach \cite{newman2015}, nach der „Microservices kleine, autonome Dienste sind, welche Zusammen Aufgaben absolvieren“. Wir analysieren im Folgenden diese drei Eigenschaften auf ihre Kernaussagen.

Zunächst handelt es sich um „kleine“ Dienste. Die Grundeigenschaft, die damit ausgedrückt werden soll, ist eine ausgeprägte \textbf{Kohäsion} der Software des Microservices. Damit wird Robert Martins \textit{Single Responsibility Principle} ausgedrückt, welcher beschreibt, dass Software, welche aus der gleichen Intention agiert, zusammengefügt werden soll und dass andere Teile der Software voneinander zu trennen sind. Das Resultat sind kleine bzw. kleinstmögliche Dienste, die einem wohlbestimmten Zweck dienen.

Die zweite geforderte Eigenschaft, die der \textbf{Autonomie}, besagt dass Microservices selbständig sind. Dies betrifft zum einen die Ausführung und Installation des Services, da er unabhängig von anderen Diensten gewartet, installiert und ausgeführt werden kann, ohne das gesamte Informationssystem zu brechen. Zum anderen betrifft es aber auch die Art und Weise, wie der Service entwickelt wird. Denn nach \cite{Fowler+14} wird ein gesamtes Entwicklungsteam entsprechend der Aufgabe aus verschiedenen Experten für Bedienoberfläche, Datenbank, DevOps o.ä. zusammengesetzt. Dementsprechend kann der Microservice auf alle erforderlichen Ressourcen zurückgreifen ohne von anderen Services abhängig zu sein.

Die Autonomie der Microservices eröffnet auch weitere Möglichkeiten für die Entwicklung. Etwa kann jeder Microservice in einer anderen Programmiersprache geschrieben werden, wenn diese dem Anwendungsfall besser dient. So kann für rechenintensive Aufgaben eine nebenläufige Sprache wie \textit{Clojure}\footnote{\url{http://clojure.org/}} verwendet werden oder es kann für einfache Aufgaben auf leichte Skriptsprachen wie etwa \textit{PHP}\footnote{\url{http://php.net/}} zurückgegriffen werden.

Zuletzt erfordert \citeauthor{newman2015}s Definition die Eigenschaft der \textbf{Kooperation}. Dies ist bedingt durch die harten Grenzen, welche ein Microservice aufwirft, die eine Kommunikation der Dienste untereinander erzwingt. Um ein gemeinsames Ziel zu erreichen, beispielsweise einen Aufgabenablauf zu erfüllen, der heterogene Teilaufgaben vorsieht, müssen die Dienste zusammen arbeiten. Die Kommunikation unter den Diensten erfolgt in festgelegten Protokollen, wie etwa \textit{Message Queueing} oder \textit{RESTful APIs}. \autoref{sec:kommunikation} expliziert, welche Protokolle dafür in Betracht gezogen werden können.

Die Verwendung abgesprochener Kommunikationsmittel indes ermöglicht das replizieren verschiedener Dienste. Denn die übertragenen Nachrichten können von einer Lastverteilung ausgelesen und auf verschiedene Knoten übertragen werden. Dadurch erreichen Microservices eine sehr hohe horizontale Skalierbarkeit, da sie quasi beliebig oft instantiiert werden können.

Nachdem wir die drei wichtigsten Eigenschaften von Microservices erläutert haben, werden wir im nächsten Abschnitt deren Prinzip von weiteren Architekturmodellen abgrenzen und motivieren.

\subsection{Abgrenzung zu anderen Modellen}
\label{subsec:abgrenzung}

Wenn man den Begriff „Microservice“ betrachtet, wird man schnell einen Bezug zu \textbf{serviceorientierten Architekturen} (SOA) erwarten. Wie bei Microservices sieht SOA eine Trennung der Anwendungslogik in funktionale Anwendungsteile vor. Diese muss aber im Gegensatz zu Microservices allerdings nicht hart sein, sondern auch innerhalb einer monolithischen Anwendungsstruktur erfolgen. Ebenso sieht SOA vor, dass die Benutzeroberfläche nicht Gegenstand der Services sei. Microservices hingegen postulieren eine Aufteilung der Oberfläche und betrachten diese als Bestandteil eines Dienstes.

Große Ähnlichkeiten weisen Microservices ebenfalls zu \textbf{Multiagentensystemen} auf. Dies ist vor allem durch die Eigenschaften eines gerichteten Zwecks, der Autonomie und der Kooperation geschuldet. Dennoch muss man beachten, dass Microservices vielmehr durch pragmatische Ansätze entstanden sind und daher die soziologisch-theoretischen Prinzipien vermissen lassen, welche die agentenorientierte Softwareentwicklung auszeichnet. Darüber hinaus finden wir in Microservices soziotechnische Eigenschaften, die wir in klassischer Agentenorientierung nicht finden. Etwa die Vorgabe von getrennten Entwicklungsteams um die Struktur der Anwendung auf die Struktur der entwickelnden Organisation zu übertragen. 

Es gibt allerdings solche Ansätze wiederum in der \textbf{organisationsorientierten Entwicklung} nach \cite{Wester-Ebbinghaus10}. Dies ist eine Weiterentwicklung des Agentenansatzes, der insbesondere die Organisationsstruktur auf die Anwendung abbildet – und umgekehrt. Microservices sehen ähnliche Synergieeffekte, sind allerdings nicht in dem Maße schachtelbar wie Organisationssysteme.

Im kommenden Unterabschnitt betrachten wir die vorgestellten Microservice-Eigenschaften mit Fokus auf Sicherheit.

\subsection{Motivation für Microservices und Sicherheit}
\label{subsec:motivation}

Aufgrund der zuvor genannten Eigenschaften lösen Microservices sowohl Herausforderungen als auch Chancen für Sicherheitsaspekte einer Architektur. Herausfordernd ist dabei das Problem, dass ein Informationssystem nur so sicher ist wie ihr schwächstes Glied; dies kann bei einer Vielzahl von kleinen Rechensystem schnell schwierig zu überblicken sein. Insbesondere die Heterogenität der Dienste kann schnell zu Ausfällen und individuellen Sicherheitslücken führen.

Auf der anderen Seite lassen sich Microservices aufgrund ihrer sehr guten Austauschbarkeit leichter warten. Für die jeweiligen Teams ist es wesentlich leichter selektiv Sicherheitslücken zu bekämpfen und neue Versionen eines Dienstes zu veröffentlichen ohne damit ein gesamtes System zu gefährden. Auch die Skalierbarkeit der Instanzen hat zur Folge das Dienstausfällen vorgebeugt werden kann. Insbesondere bei einer DoS-Attacke können z.B. durch das schnelle Beenden und Neustarten von Microservices Gegenmaßnahmen ergriffen werden.

Nachdem wir mit diesem Abschnitt das Microservice-Konzept vorgestellt haben, diskutiert der folgende Microservice-Architekturen und legt dabei seinen Schwerpunkt auf die Sicherheitseigenschaften dieser.
\section{Microservice-Architekturen und Sicherheit}
\section{Kommunikation und Sicherheit}

Wie in Kapitel \ref{sec:ms-arch-sec} beschrieben, besteht ein Microservice-basiertes System aus einer Menge von mindestens einem physikalischen Host und beliebig vielen Microservices. Um überhaupt von einem System sprechen zu können, müssen die einzelnen Komponenten noch vernetzt werden. Dazu benutzen wir im Folgenden den Ausdurck \textit{Service-to-Service-Kommunikation} -- StS-Kommunikation. Services kommunizieren miteinander, tauschen Nachrichten aus, fordern Dinge an die von einem anderen Service berechnet werden müssen etc \cite{newman2015}. Diese Kommunikation kann angegriffen werden, insbesondere wenn die Kommunikation über unsichere Kanäle stattfindet wie etwa das World-Wide-Web.

Im Folgenden wollen wir verschiedene Möglichkeiten von \stscom\ erläutern, Vor- und Nachteile diskutieren und auf Sicherheitsaspekte eingehen.

\subsection{Kommunikationswege}

Es gitb verschiedene klassische Ansätze, Software-Komponenten oder Services miteinander kommunizieren zu lassen.


\subsection{Zusammenfassung}

In diesem Kapitel haben wir verschiedene Möglichkeiten zur \stscom\ aufgezeigt. Dabei wurden insbesondere REST und Message-Queues betrachtet. Mögliche Schwachstellen bei der \stscom\ wurden herausgestellt und der Bedarf nach applikationsseitigem Schutz wurde deutlich gemacht. Im Folgenden werden wir Schutzziele definieren und auf Systeme sowie auf Kommunikation übertragen. Es werden einige konkrete Verfahren und Bibliotheken vorstellen, mit denen sich diese Schutzziele umsetzen lassen. 
\section{Schutzziele}
Nachdem wir bisher auf die Architektur und die Kommunikation von Microservices eingegangen sind, betrachten wir im folgenden Abschnitt Angriffspunkte einer solchen Infrastruktur. Explizit werden drei Schutzziele der Informationssicherheit betrachtet, welche eine besondere Relevanz in verteilten Systemen spielen. Zunächst gehen wir auf das der Vertraulichkeit in \autoref{subsec:vertraulichkeit} ein. Weitergehend setzt sich \autoref{subsec:verfügbarkeit} mit der Verfügbarkeit und zuletzt \autoref{subsec:integrität} mit der Integrität von Informationen innerhalb eines Microservice-Netzwerks auseinander.


\subsection{Vertraulichkeit}
\label{subsec:vertraulichkeit}
„Informationsvertraulichkeit ist bei einem IT-System gewährleistet, wenn die darin enthaltenen Informationen nur Befugten zugänglich sind. Dies bedeutet, dass die sicherheitsrelevanten Elemente nur einem definierten Personenkreis bekannt werden. Dazu sind Maßnahmen zur Festlegung sowie zur Kontrolle zulässiger Informationsflüsse zwischen den Subjekten des Systems nötig (Zugriffsschutz und Zugriffsrechte), sodass ausgeschlossen werden kann, dass Informationen zu unautorisierten Subjekten ‚durchsickern‘.“ \cite{Bedner+10}

\subsubsection{Handshake-basierte Authentifikation}

\subsubsection{Token-basierte Authentifikation}


\subsection{Verfügbarkeit}
\label{subsec:verfügbarkeit}
„Die Verfügbarkeit betrifft sowohl informationstechnische Systeme als auch die darin verarbeiteten Daten und bedeutet, dass die Systeme jederzeit betriebsbereit sind und auf die Daten wie vorgesehen zugegriffen werden kann. Zum einen muss die Datenverarbeitung inhaltlich korrekt sein und zum anderen müssen alle Informationen und Daten zeitgerecht zur Verfügung stehen und ordnungsgemäß verarbeitet werden.“ \cite{Bedner+10}

\subsubsection{Redundanz und Lastverteilung}

\subsubsection{Client-seitige Lastverteilung}

\subsubsection{Kontinuierliches und unabhängiges Deployment}


\subsection{Integrität}
\label{subsec:integrität}
„Integrität oder Unversehrtheit bedeutet zweierlei, nämlich die Vollständigkeit und Korrektheit der Daten (Datenintegrität) und die korrekte Funktionsweise des Systems (Systemintegrität). Vollständig bedeutet, dass alle Teile der Information verfügbar sind. Korrekt sind Daten, wenn sie den bezeichneten Sachverhalt unverfälscht wiedergeben. Die Integrität bedeutet, dass Daten im Laufe der Verarbeitung oder Übertragung mittels des Systems nicht beschädigt oder durch Nichtberechtigte unbefugt verändert werden können. Beschädigungs- oder Veränderungsmöglichkeiten sind das Ersetzen, Einfügen und Löschen von Daten oder Teilen davon.“ \cite{Bedner+10}

\subsubsection{Verschlüsselungsverfahren}
\section{Zusammenfassung}
\label{sec:zsmfassung}


\nocite{*}
\bibliographystyle{apacite}
\bibliography{bib}

\endgroup
\end{document}
